%%%%%%%%%%%%%%%%%%%%%%% file template.tex %%%%%%%%%%%%%%%%%%%%%%%%%
%
% This is a general template file for the LaTeX package SVJour3
% for Springer journals.          Springer Heidelberg 2010/09/16
%
% Copy it to a new file with a new name and use it as the basis
% for your article. Delete % signs as needed.
%
% This template includes a few options for different layouts and
% content for various journals. Please consult a previous issue of
% your journal as needed.
%
%%%%%%%%%%%%%%%%%%%%%%%%%%%%%%%%%%%%%%%%%%%%%%%%%%%%%%%%%%%%%%%%%%%
%
% First comes an example EPS file -- just ignore it and
% proceed on the \documentclass line
% your LaTeX will extract the file if required
\begin{filecontents*}{example.eps}
%!PS-Adobe-3.0 EPSF-3.0
%%BoundingBox: 19 19 221 221
%%CreationDate: Mon Sep 29 1997
%%Creator: programmed by hand (JK)
%%EndComments
gsave
newpath
  20 20 moveto
  20 220 lineto
  220 220 lineto
  220 20 lineto
closepath
2 setlinewidth
gsave
  .4 setgray fill
grestore
stroke
grestore
\end{filecontents*}
%
\RequirePackage{fix-cm}
%
%\documentclass{svjour3}                     % onecolumn (standard format)
%\documentclass[smallcondensed]{svjour3}     % onecolumn (ditto)
\documentclass[smallextended]{svjour3}       % onecolumn (second format)
%\documentclass[twocolumn]{svjour3}          % twocolumn
%
\smartqed  % flush right qed marks, e.g. at end of proof
%
\usepackage{graphicx}
%
% \usepackage{mathptmx}      % use Times fonts if available on your TeX system
%
% insert here the call for the packages your document requires
%\usepackage{latexsym}
% etc.
%
% please place your own definitions here and don't use \def but
% \newcommand{}{}
%
% Insert the name of "your journal" with
% \journalname{myjournal}
%
\begin{document}

\title{Titleless%\thanks{Grants or other notes
%about the article that should go on the front page should be
%placed here. General acknowledgments should be placed at the end of the article.}
}
%\subtitle{Do you have a subtitle?\\ If so, write it here}

\titlerunning{Assistive Devices}        % if too long for running head

\author{Ramele Rodrigo        \and
        Santos Juan Miguel %etc.
}

%\authorrunning{Short form of author list} % if too long for running head

\institute{F. Author \at
              first address \\
              Tel.: +123-45-678910\\
              Fax: +123-45-678910\\
              \email{fauthor@example.com}           %  \\
%             \emph{Present address:} of F. Author  %  if needed
           \and
           S. Author \at
              second address
}

\date{Received: date / Accepted: date}
% The correct dates will be entered by the editor


\maketitle

\begin{abstract}

Background / introduction
Methods
Results
Conclusions

Cognitive computing platforms are pervasive.  However in terms of desing is hard to detect.  How alpha waves determine the inhibition process in the Brain?  An idle common frequency between different components, which is emphasized by an equivalent alpha wave signal, which is exactly diminished when activity starts to be detected.

Devices have entered human environment, but mainly in the form of information providers, or contextual drivers.  Yet sensing and action is still absence.  Intelligent devices, that sense and act accordingly are still just in dream form.

Internet of Robotics Things is a field on the intersection of Internet of Things and Robotics.


Yet this coming of age of mobile devices, is not through direct inmersion of robotics, it is the development of cheap and interconnected devices.  Robotics will come, in hands of the internet of things.

In this work we propose a very simple and yet powerful architecture which is fully based on IoT devices.


The general idea mains three direct connections 

Communications
Security
Power> this is perhaps the most complex and it can get enough from industry4.0

we show that a standard architecture can be achieved using low cost components.

Furthermore, the proposed architecture get inspiration from Neurosciences study by mimicking how oscilatory signals play an important role in the synchronization of different areas of the mammalian cortex.  This procedure is replicated to imitate and solve, the real time synchronization of different components of a robotic application.   This proposed architecture is studied by performing five traditional benchmark operations.  



\keywords{First keyword \and Second keyword \and More}
% \PACS{PACS code1 \and PACS code2 \and more}
% \subclass{MSC code1 \and MSC code2 \and more}
\end{abstract}

\section{Introduction}
\label{intro}
Unreliability of cheap sensors


Internet of Robotic Things \cite{Simoens2018}.

Simulating the predictability of RTOS using inspiration from alpha waves.

Multicores NVIDEA 

SIM Particle

Reliability of SIM cards.

Long range wifi 802.11n

How it can be implemented without any need to add extra middleware, just by using out of the box python programs and tools.  Other works like pyro

The proposed mechanism entails using USB Hubs as standard connectors.  This is particularly interesting with USB 3.0 which provides more power and faster processing speed, while using a standard and cheap interconnection method.

For third world countries where cheap is necesity 

The difference between robotics and automation is that the former is a general purpose automaton machine while the latter just generates repetitive taks.

Robotics in third world countries and developing economies.  Opportunity and risks.

Magnetic sensors to establish indoor localization (REF PAPER SYDNEY).

A testbed laboratory based on robotics \cite{Hamblen2013}.


Middlware-less solution, where the middleware is not strictly necessary.
\section{Methods}
\label{sec:1}
Any basic robotic technology requires to sense, to compute and to act.  
Any robotic solution requires different set of heterogeneous tools, which involve sensors, actuators, computes and so on.  Putting all toghether requires middleware that allow to transmit effectively messages between different parts of the system.
The work offered by \cite{Elkady2012} provides an extensive survey of different robotic frameworks.  This works describes a kiss middleware framework based on python and serial connections.   On one hand, python is an extensive language which is now widespread usage on machine learning solutions.  On the other hand, serial connection is an old but very robuts connection interface.

The general diagram can be outlined in Figure .

Any IoRT must have the following technical capacities

\begin{itemize}
\item Initial Connection: captive portal
\item Security
\item Starting up
\item Networking
\item Auto testing and diagnosis
\item Telemetry
\item Streaming
\item Software updating
\end{itemize}

Once the connection is finally established, the following affordances should be obtained

\begin{itemize}
\item Planar orientation: which is my orientation in relation to a previous orientation.
\item Localization: it can be achieved externally and internally based on planar orientation and displacement tracking.
\item Mapping can be achieved through the same means by using the combination of planar orientation, localization and external sensors.
\end{itemize}


\begin{itemize}
\item Foraging: the first benchmark can be achieved on the top of the former.
\item Human-Robot Collaboration
\item Robotic Collaboration
\item Semantic Mapping
\end{itemize}

\subsection{Real Time Microcontrolers and Real time operating systems}

Let's say we need 15 Hz decoding 

\subsection{Security}

The system provides that the entity that achieved the captive portal initial authentication will have the tools to establish a secure trusted and confidential communication with the robotic platform.

Offer a threat model.  Link with IoT Security.

\subsection{Power}

Cheap and accesible power solutions are mandatory for service robots.  Power banks have all that and more.
However power electronics are necessary besides power banks to provide.

\subsection{Benchmarks}

Existing infrastructure
Telemetry
Streaming
Initial kick off and autoconnect using a portal implemented in that and that way

Include network latency.  How well and timing and for how long is able to reply back.

\section{Results}
\label{sec:results}

Similar to ROSSerial a driving connection to drive out the buffering issue with serial communication between different components.

\subsection{Heading Levels}

Results for the benchmarks are shown in Figures this and that.  It can be seen how the robotic operated under 


Components 

sensors

Modules

Software components and architecute

Block diagram

Battery Management System 
\section{Discussion}

\section{Summary}

They require an extra power unit to handle power for the electronics components as well as the power electronics for motors or other devices.  Without such a unit the ability of the robot to handle their own power supply is diminished.


alpha dissapears when alerting by any mechanism (thinking, calculating)

\paragraph{Paragraph headings} Use paragraph headings as needed.
\begin{equation}
a^2+b^2=c^2
\end{equation}

% For one-column wide figures use
\begin{figure}
% Use the relevant command to insert your figure file.
% For example, with the graphicx package use
  \includegraphics{example.eps}
% figure caption is below the figure
\caption{Please write your figure caption here}
\label{fig:1}       % Give a unique label
\end{figure}
%
% For two-column wide figures use
\begin{figure*}
% Use the relevant command to insert your figure file.
% For example, with the graphicx package use
  \includegraphics[width=0.75\textwidth]{example.eps}
% figure caption is below the figure
\caption{Please write your figure caption here}
\label{fig:2}       % Give a unique label
\end{figure*}
%
% For tables use
\begin{table}
% table caption is above the table
\caption{Please write your table caption here}
\label{tab:1}       % Give a unique label
% For LaTeX tables use
\begin{tabular}{lll}
\hline\noalign{\smallskip}
first & second & third  \\
\noalign{\smallskip}\hline\noalign{\smallskip}
number & number & number \\
number & number & number \\
\noalign{\smallskip}\hline
\end{tabular}
\end{table}


\begin{acknowledgements}
If you'd like to thank anyone, place your comments here
and remove the percent signs.
\end{acknowledgements}

% BibTeX users please use one of
\bibliographystyle{spbasic}      % basic style, author-year citations
%\bibliographystyle{spmpsci}      % mathematics and physical sciences
%\bibliographystyle{spphys}       % APS-like style for physics
\bibliography{cogn}   % name your BibTeX data base

% Non-BibTeX users please use
\begin{thebibliography}{}
%
% and use \bibitem to create references. Consult the Instructions
% for authors for reference list style.
%
\bibitem{RefJ}
% Format for Journal Reference
Author, Article title, Journal, Volume, page numbers (year)
% Format for books
\bibitem{RefB}
Author, Book title, page numbers. Publisher, place (year)
% etc
\end{thebibliography}

\end{document}
% end of file template.tex

